Parte final do texto, na qual são apresentadas conclusões correspondentes aos objetivos e/ou às hipóteses. É o fecho do trabalho. Nessa parte, explicitamos a resposta à pergunta do problema de investigação, bem como possíveis limitações do estudo.

A conclusão deve ser breve. Visa a recapitular, sinteticamente, os resultados da pesquisa feita, evidenciando qual ou quais hipótese(s) do trabalho se confirma(m) e o porquê.

Ao escolher um tema para trabalhar, é preciso que o pesquisador faça um
inventário do conhecimento disponível e proceda a uma triagem daquilo que pode ser útil para explicar a nova situação proposta.

Nem sempre uma conclusão é uma resposta final e acabada a um problema.
Ao contrário, boas conclusões devem deixar “portas abertas” para novas propostas de pesquisa em torno do tema estudado, além de evidenciar que contribuições o estudo proporcionou no âmbito acadêmico, no profissional e para a sociedade.

Devem ser apontadas as dificuldades que tenham sido responsáveis ou por limitar o alcance das conclusões do estudo, ou por determinar opções de trabalho, ou qualquer outra que tenha contribuído para dar cunho particular ao estudo, dificuldades essas que poderão, inclusive, ser revistas em trabalhos futuros.