Este modelo de documento foi criado seguindo as especificações do Manual de Metodologia Científica \cite{prodanov}. Para isto foram criados vários estilos, com a formatação apropriada para cada situação. No texto são utilizados exemplos para cada um dos estilos. Caso feito utilizando o Word, o sumário é criado automaticamente e deve ser atualizado com o clique direito do mouse sobre ele e em seguida "Atualizar Campo" no menu suspenso que aparece. Utilizando \LaTeX esse processo é feito automaticamente ao compilar o projeto.

Conforme a NBR 14724, da \citeonline{NBR14724:2011} a introdução, parte inicial do texto, é onde devem constar a delimitação do assunto tratado, os objetivos da pesquisa e outros elementos necessários para situar o tema do trabalho\footnote{Veja um exemplo de nota de rodapé}. A introdução refere-se ao posicionamento da questão central da Monografia, ou seja, da colocação clara do problema de pesquisa, dos objetivos do trabalho, bem como dos meios a serem utilizados para tal. Deve incluir, também, a justificativa de escolha do tema, o que constitui fator importante para avaliação do critério utilizado na seleção dos dados trabalhados.
Deve ser sintética e sua extensão é proporcional ao porte do trabalho. É, do ponto de vista lógico, a primeira parte que o leitor encontrará e a última a ser escrita pelo pesquisador.
A Introdução deve incluir:
\begin{itemize}
    \item o tema da monografia e a justificativa de sua escolha; a relevância e as contribuições para a área em que se insere;
    \item o problema de pesquisa;
    \item a hipótese estabelecida;
    \item o objetivo geral e os objetivos específicos do trabalho. Também são apresentados os procedimentos metodológicos básicos (métodos, técnicas, instrumento de coleta de dados etc.) e o quadro-teórico empregado, relacionando-o ao objeto de estudo. Além disso, serão informadas, de forma sintética, as partes que compõem o trabalho, indicando, por exemplo, que no \autoref{chap:capitulo-exemplo} serão abordados mais exemplos de formatação.
\end{itemize}