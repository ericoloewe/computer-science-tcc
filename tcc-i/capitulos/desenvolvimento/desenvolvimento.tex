Segundo a NBR 14724, é parte principal do texto, que contém a exposição
ordenada e pormenorizada do assunto. Divide-se em seções e subseções, que variam em função da abordagem do tema e do método.

Essa parte do trabalho deve incluir o processo de explicação do problema central da Monografia (o objeto de estudo ou o Problema de Investigação, se usarmos linguagem de Pesquisa), das hipóteses de trabalho e das técnicas utilizadas para obter dados, verificando, assim, as hipóteses elaboradas.

É extremamente importante, nessa parte, que nos guiemos por uma firme
orientação metodológica. Será a metodologia escolhida e empregada o elemento definidor da qualidade do trabalho. Através da metodologia, podemos não apenas concluir, como também comprovar por que as conclusões a que chegamos são válidas e consistentes.
Em síntese, DESENVOLVIMENTO representa os capítulos do trabalho e seus títulos, subtítulos, itens e subitens criados pelo autor, devendo manter relação direta com o tema e lógica entre si. Deve conter a exposição ordenada e pormenorizada do assunto. Divide-se em seções e subseções, que variam em função da forma de abordagem dada ao tema. Pode conter material explicativo e ilustrativo (quadros, gráficos, tabelas, fotos etc.). No caso da tese (doutorado), devemos escrever um
capítulo argumentando, explicando e demonstrando a tese comprovada e, em seguida, fazer a relação entre ele e os demais.
Partes que integram o desenvolvimento são abordadas a seguir.

\subfile{capitulos/desenvolvimento/revisao-literatura}
\subfile{capitulos/desenvolvimento/metodologia}
\subfile{capitulos/desenvolvimento/resultados}
