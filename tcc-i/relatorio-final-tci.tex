%
% MODELO LATEX PARA TRABALHOS ACADÊMICOS DA FEEVALE NO CURSO DE CIÊNCIA DA COMPUTAÇÃO
%
% INSTRUÇÕES GERAIS:
%    1. O PROJETO POSSUI ARQUIVOS ".tex",  NO QUAL O TRABALHO SERÁ ESCRITO, E O ARQUIVO ".bib", NO QUAL SERÃO DEFINIDAS AS REFERÊNCIAS BIBLIOGRÁFICAS. OS DEMAIS SÃO ARQUIVOS ANEXOS DO TRABALHO OU PACOTES AUXILIARES
%    2. O TEXTO APÓS O CARACTERE'%' É COMENTÁRIO, NÃO FAZ DIFERENÇA NO RESULTADO FINAL 
%    3. O PROJETO ESTÁ SEPARADO EM ARQUIVOS INDIVIDUAIS POR CAPÍTULO, TODOS DEVEM ESTAR INFORMADOS NO FINAL DESTE DOCUMENTO
%    4. NESTE DOCUMENTO SÃO CONFIGURADAS AS DEFINIÇÕES PRINCIPAIS DO TRABALHO, COMO NOME DO ALUNO, TÍTULO, ETC, ALÉM DE CONFIGURAÇÕES DE FORMATAÇÃO, QUE NÃO SERÁ NECESSÁRIO ALTERAR
%    5. AS INFORMAÇÕES ESPECÍFICAS DO TRABALHO PODEM SER DESCRITAS A PARTIR DA MARCAÇÃO "\titulo"

\documentclass[
	% -- opções da classe memoir --
	12pt,				% tamanho da fonte
	% openright,		% capítulos começam em pág ímpar (insere página vazia caso preciso)
    oneside,			% para impressão somente frente. Oposto a twoside (frente e verso)
	a4paper,			% tamanho do papel 
	% -- opções da classe abntex2 --
	%chapter=TITLE,		% títulos de capítulos convertidos em letras maiúsculas
	%section=TITLE,		% títulos de seções convertidos em letras maiúsculas
	%subsection=TITLE,	% títulos de subseções convertidos em letras maiúsculas
	%subsubsection=TITLE,% títulos de subsubseções convertidos em letras maiúsculas
	% -- opções do pacote babel --
	english,			% idioma adicional para hifenização
	french,				% idioma adicional para hifenização
	spanish,			% idioma adicional para hifenização
	brazil,				% o último idioma é o principal do documento
	]{abntex2-modificado}


% ---
% PACOTES
% ---

% ---
% Pacotes fundamentais 
% ---
\usepackage{cmap}				% Mapear caracteres especiais no PDF
\usepackage{lmodern}			% Usa a fonte Latin Modern
\usepackage[T1]{fontenc}		% Seleção de códigos de fonte
\usepackage[utf8]{inputenc}		% Codificação do documento (conversão automática dos acentos)
\usepackage{indentfirst}		% Indenta o primeiro parágrafo de cada seção
\usepackage{color}				% Controle das cores
\usepackage{graphicx}			% Inclusão de gráficos
% ---

% ---
% Pacotes adicionais, usados no anexo do modelo de folha de identificação
% ---
\usepackage{multicol}
\usepackage{multirow}
% ---
	
% ---
% Pacotes adicionais, usados apenas no âmbito do Modelo Canônico do abnteX2
% ---
\usepackage{lipsum}				% para geração de dummy text
% ---

% ---
% Pacotes de citações
% ---
\usepackage[brazilian,hyperpageref]{backref}	 % Paginas com as citações na bibl
\usepackage[alf]{abntex2cite}	% Citações padrão ABNT

\usepackage{subfiles}
\let\newfloat\undefined
\usepackage[capposition=top]{floatrow}
\usepackage{booktabs}
% ---
% Pacotes para pseudocódigos
% ---
\usepackage{listings} %For code in appendix
\lstset
{ %Formatting for code in appendix
    basicstyle=\footnotesize,
    numbers=left,
    stepnumber=1,
    showstringspaces=false,
    tabsize=1,
    breaklines=true,
    breakatwhitespace=false,
    xleftmargin=1cm,
    xrightmargin=1cm,
    %frame=single,
    keywords={,Função,se,então,senão,fimse,para,cada,em,fimpara,faça,retorne,nulo,enquanto,fimenquanto, },
    literate=
        {á}{{\'a}}1
        {à}{{\`a}}1
        {ã}{{\~a}}1
        {é}{{\'e}}1
        {ê}{{\^e}}1
        {í}{{\'i}}1
        {ó}{{\'o}}1
        {õ}{{\~o}}1
        {ú}{{\'u}}1
        {ü}{{\"u}}1
        {ç}{{\c{c}}}1
        {:=}{$\leftarrow$}1
        {:delta}{$\delta$}1
        {:tau}{$\tau$}1
        {:epsilon}{$\epsilon$}1
}
% --- 
% CONFIGURAÇÕES DE PACOTES
% --- 

% ---
% Configurações do pacote backref
% Usado sem a opção hyperpageref de backref
\renewcommand{\backrefpagesname}{Citado na(s) página(s):~}
% Texto padrão antes do número das páginas
\renewcommand{\backref}{}
% Define os textos da citação
\renewcommand*{\backrefalt}[4]{
	\ifcase #1
		Nenhuma citação no texto.
	\or
		Citado na página #2.
	\else
		Citado #1 vezes nas páginas #2.
	\fi}
% ---

% ---
% Informações de dados para CAPA e FOLHA DE ROSTO
% ---
\titulo{TÍTULO DO TRABALHO}
\autor{NOME DO ALUNO}
\orientador{Nome Do Orientador}
\local{Novo Hamburgo}
\data{20xx}
\instituicao{Universidade Feevale}
\tipotrabalho{Trabalho de Conclusão de Curso}
% O preâmbulo deve conter o tipo do trabalho, o objetivo, o nome da instituição e a área de concentração
\preambulo{Trabalho de Conclusão de Curso apresentado como requisito parcial à obtenção do grau de Bacharel em Ciência da Computação pela Universidade Feevale}
% ---

% ---
% Configurações de aparência do PDF final
% ---
% alterando o aspecto da cor azul
\definecolor{blue}{RGB}{41,5,195}
\definecolor{black}{RGB}{0,0,0}

% Informações do PDF
\makeatletter
\hypersetup{
     	%pagebackref=true,
		pdftitle={\@title}, 
		pdfauthor={\@author},
    	pdfsubject={\imprimirpreambulo},
	    pdfcreator={},
		pdfkeywords={}, 
		colorlinks=true,      	% false: boxed links; true: colored links
    	linkcolor=black,        % color of internal links
    	citecolor=black,        % color of links to bibliography
    	filecolor=magenta,      % color of file links
		urlcolor=black,
		bookmarksdepth=4
}
\makeatother
% --- 

% --- 
% Espaçamentos entre linhas e parágrafos 
% --- 
% O tamanho do parágrafo é dado por:
\setlength{\parindent}{1.3cm}

% Controle do espaçamento entre um parágrafo e outro:
\setlength{\parskip}{0.2cm}  % tente também \onelineskip

% ---
% compila o índice
% ---
\makeindex
% ---

% ----
% Início do documento
% ----
\begin{document}

% Retira espaço extra obsoleto entre as frases.
\frenchspacing 

% ----------------------------------------------------------
% ELEMENTOS PRÉ-TEXTUAIS
% ----------------------------------------------------------
% \pretextual

% ---
% Capa
% ---
\imprimircapa
% ---

% ---
% Folha de rosto
% (o * indica que haverá a ficha bibliográfica)
% ---
\imprimirfolhaderosto*
% ---

% ---
% Folha de aprovação
% ---
\begin{folhadeaprovacao}
  \begin{center}
    {\ABNTEXchapterfont\large\MakeTextUppercase\imprimirautor}
    \vspace*{\fill}\vspace*{\fill}
    \begin{center}
      \ABNTEXchapterfont\large\MakeTextUppercase\imprimirtitulo
    \end{center}
    \vspace*{\fill}
    \hspace{.45\textwidth}
    \begin{minipage}{.5\textwidth}
      \imprimirpreambulo
    \end{minipage}
    \vspace*{\fill}
  \end{center}
  APROVADO EM: \textbf{\_\_} / \textbf{\_\_} / \textbf{\_\_\_\_}
  \assinatura{\MakeTextUppercase{\imprimirorientador} \\ Orientador – Feevale}
  \assinatura{\MakeTextUppercase{NOME DO AVALIADOR 1} \\ Examinador interno – Feevale}
  \assinatura{\MakeTextUppercase{NOME DO AVALIADOR 2} \\ Examinador interno – Feevale}
  \begin{center}
    \vspace*{0.5cm}
    {\large\imprimirlocal}
    \par
    {\large\imprimirdata}
    \vspace*{1cm}
  \end{center}
\end{folhadeaprovacao}
% ---

% ---
% AGRADECIMENTOS
% ---
% A folha de dedicatória ou de agradecimentos é um elemento opcional
\begin{agradecimentos}
\begin{vplace}[0.5]
\hspace{.45\textwidth}
\begin{minipage}{.5\textwidth}
Gostaria de agradecer a todos os que, de alguma maneira, contribuíram para a realização desse trabalho de conclusão, em especial: 
Aos amigos e às pessoas que convivem comigo diariamente, minha gratidão, pelo apoio emocional - nos períodos mais difíceis do trabalho.  
Enfim, os demais agradecimentos que o aluno desejar fazer.
\end{minipage}
\end{vplace}
\end{agradecimentos}
% ---

% ---
% RESUMO
% ---
% Resumo na língua vernácula e estrangeira (obrigatório)
\begin{resumo}
 \noindent
 O resumo deve apresentar, de forma breve, o tema e sua importância, os objetivos, o marco teórico principal, a metodologia e os resultados alcançados, ou seja: os pontos relevantes do texto, fornecendo uma visão rápida e clara do conteúdo e das conclusões do trabalho. O resumo deve ser formado por frases concisas, não utilizando enumeração de tópicos. Assim como em todo o restante do trabalho, termos em língua estrangeira devem ser destacados em itálico, como por exemplo o termo \textit{abstract}. O resumo deve conter entre 150 e 500 palavras ao todo.

 \vspace{\onelineskip}
    
 \noindent
 %Selecionar 3 a 5 palavras chave para o trabalho
 Palavras-chave:  Modelo de TC. LaTeX. Universidade Feevale.
\end{resumo}

%Resumo em língua estrangeira (obrigatório)
\begin{resumo}[Abstract] 
\begin{otherlanguage*}{english}
 \noindent
 \textit{Resumo escrito em língua estrangeira, contendo as mesmas informações e características do resumo na língua vernácula. Por se tratar da língua estrangeira o texto deve ser escrito todo em itálico}

 \vspace{\onelineskip}
    
 \noindent
 \textit{Keywords: Final Paper Model. LaTeX. Feevale University.}
\end{otherlanguage*}
\end{resumo}

% ---
% Lista de ilustrações
% ---

\listoffigures* % o caractere * indica que não será incluso no sumário
% A lista será gera automaticamente conforme as imagens forem utilizadas no restante do trabalho
\cleardoublepage % Pula página
% ---

% ---
% Lista de tabelas
% ---

\listoftables*
% A lista será gera automaticamente conforme as tabelas forem utilizadas no restante do trabalho
\cleardoublepage
% ---

% ---
% Lista de Abreviaturas (Opcional)
% ---
% As siglas devem ser citadas em ordem alfabética
\begin{siglas}
 \item[ABNT] Associação Brasileira de Normas Técnicas
 \item[NBR] Norma Brasileira aprovada pela ABNT
 \item[ICCT] Instituto de Ciências Criativas e Tecnológicas
 \item[ICS] Instituto de Ciências da Saúde
 \item[ICHS] Instituto de Ciências Humanas e Sociais
 \item[TCC] Trabalho de Conclusão de Curso
 \item[URL] \textit{Uniform Resource Locator}
\end{siglas}

% ---
% Sumário
% ---
\tableofcontents*
%\thispagestyle{empty}
\cleardoublepage
% ---

% ----------------------------------------------------------
% ELEMENTOS TEXTUAIS  (necessário para incluir número nas páginas)
% ----------------------------------------------------------
% Abaixo está montada a estrutura do TC. Sugere-se separar os capítulos em arquivos distintos
\textual

\chapter{Introdução}
% O marcador chap: define o nome interno do capítulo neste documento. Será utilizado no restante do trabalho quando for necessário citar o capítulo específico
\label{chap:introducao}
\subfile{capitulos/introducao/introducao}

\chapter{Capítulo de exemplo}
\label{chap:capitulo-exemplo}
\subfile{capitulos/desenvolvimento/desenvolvimento}
\label{chap:capitulo-exemplo2}
\subfile{capitulos/exemplos/exemplos}

\chapter{Conclusão}
\label{chap:conclusao}
\subfile{capitulos/conclusao/conclusao}


% ----------------------------------------------------------
% ELEMENTOS PÓS-TEXTUAIS
% ----------------------------------------------------------
\postextual

% ----------------------------------------------------------
% Referências bibliográficas
% ----------------------------------------------------------
\bibliography{referencias-bibliograficas}

\end{document}
