\documentclass{article}
\usepackage{graphicx,url}
\usepackage{amsmath}
\usepackage{listings}
\usepackage[brazil]{babel}   
\usepackage[utf8]{inputenc}

\title{SISTEMAS DE RECOMENDAÇÃO MUSICAL BASEADO EM CONTEXTO COMPORTAMENTAL E DE AMBIENTE}
\author{ÉRICO DE SOUZA LOEWE}

\date{Today}

\begin{document}

\makeatletter
\begin{titlepage}
   \vspace*{\stretch{1.0}}
   \begin{center}
      \large\textbf{
      UNIVERSIDADE FEEVALE\\
      \@author\\
      \@title\\
      }
      (Título Provisório)\\
      Anteprojeto de Trabalho de Conclusão\\
      \today
   \end{center}
   \vspace*{\stretch{2.0}}
\end{titlepage}
\makeatother

\section{Folha de rosto}
\newpage

\section{Resumo}
\newpage

\section{Motivação}

% -- Conceito
A tecnologia avançou muito nos últimos anos, principalmente quando estamos falando de internet e armazenamento. \cite{muraro2009avanccos} 
O custo de armazenar um arquivo vem cada vez ficando mais barato e tudo isso, tem feito com que as pessoas tenham mais espaço de armazenando, dando a possibilidade de gerarem cada vez mais informações. \cite{ufc2020amagnetorresistencia} 
A quantidade de aplicações disponíveis na internet tem aumentado cada vez mais gerando cada vez mais dados e opções para os usuários. % citar  

Diversas vezes o individuo possui dificuldades em realizar escolhas entre as diversas alternativas que lhe é apresentado, e acaba geralmente cofiando nas escolhas que lhe são apresentadas através de outras pessoas. \cite{resnick1997recommender} A partir do aumento da quantidade de informações disponíveis e do conhecimento da habilidade do individuo de realizar escolhas a partir de sua experiencia pessoal surgem os sistemas de recomendação que buscam filtrar a grande massa de dados disponíveis, para auxiliar o individuo na escolha das opções disponíveis.

Sistemas de recomendação (RecSys) são implementações de softwares e técnicas, que apresentam sugestões de itens que seriam de uso de um usuário. As sugestões são de acordo com vários processos de decisão, como, que item comprar, que musica escutar ou que noticia ler.  \cite{ricci2011introduction}

O auxilio que um sistemas de recomendação prove pode ser bem especifico ou genérico. Isso vai depender do tipo de filtragem escolhida para realizar a recomendação. Quando um sistema busca uma filtragem que leva em consideração a preferencias do usuário, elas podem ser obtidas implicitamente através de um monitoramento de comportamento. Mas também, um sistema de recomendação pode obter explicitamente sua preferencia através de perguntas sobre suas preferencias.
\cite{jannach2010recommender}

As recomendações personalizadas necessitam que o sistema conheça algo sobre cada usuário da base.
Todo sistema de recomendação deve desenvolver e manter um \textit{user model} ou \textit{user profile}, que por exemplo, contem as preferencias do mesmo.
A existência de um \textit{user model} é essencial para qualquer sistema de recomendação. \cite{jannach2010recommender}

\cite{ricci2011introduction} traz em sua obra os 4 tipo de sistemas de recomendação, sendo eles, recomendação colaborativa, recomendação baseada em conteúdo, recomendação baseada em conhecimento, e sistemas de recomendação híbridos.

Os sistemas de recomendação tiveram o seu inicio com a "Usenet" da Duke University na década de 70. \cite{bhatnagar2016collaborative} Desenvolvida em um formato cliente servidor permitia que os usuários classificassem a entrada em específicos grupos ("\textit{newsgroups}").

----- \\

Sistemas de Recomendação
   conceito
   histórico
   Exemplos de uso
   Sistemas de Recomendação para Streamming
   Vantagens/desvantagens
----- \\

% devo contar historia sobre porque o recsys é importante? (devido a grande quantidade de dados) exemplos (fora da musica) ✔
% quando o Recsys comecou? 
% como RecSys revolucionou o mercado? 

% Devo explicar sobre os 4 tipos de sistemas?

% falar sobre concurso netflix 2006 https://en.wikipedia.org/wiki/Netflix_Prize

% que tipos de aplicações existem que usam do RecSys?


----- \\

Serviços de Streamming Musicais
  conceito
  histórico
  exemplos
  mais utilizado a partir de alguma estatística
----- \\

% onde é utilizado?

% existe alguma particularidade nesses sistemas para musicas?

% quantidade de pessoas que utilizam do streaming musical

% como o streaming musical revolucionou a area

% falar sobre historia do spotify

% quantidade de usuários

O que faltam nos sistemas de recomendação musicais para serviços de streamming

% de que maneira o spotify monta as playlists (TCC 1)

Formula a tua questão de pesquisa

% como que vamos buscar melhorar esse serviço?

Objetivo Geral

%??

Atualmente, os Spotify não consegue captar um sentimento tão bem, devido a não levar em consideração as ações que os usuários estão tomando naquele momento. 

A ideia desse projeto é que com ele, possamos evoluir a recomendação musical para algo mais voltado ao sentimento que o usuário esta sentindo no momento em que esta ouvindo a musica.

\newpage

\section{Objetivos}

\subsection{Objetivo geral}

Desenvolver um sistema de recomendação musical, considerando o contexto comportamental do usuário, bem como o contexto do ambiente onde ele encontra-se.

\subsection{Objetivo específicos}

\begin{itemize}
\item Investigar APIs de Serviços de \textit{Streammings} Musicais.

\item Selecionar a API a ser utilizada no sistema de recomendação.

\item Criar a infraestrutura necessária para o armazenamento e relacionamento das músicas com os contextos comportamentais e de ambiente do usuário.

\item Avaliar o sistema de recomendação com usuários voluntários.

\end{itemize}

\newpage

\section{Metodologia}

\newpage

\section{Cronograma}

\newpage

\bibliographystyle{sbc}
% \bibliographystyle{acm}
\bibliography{references}

\end{document}
