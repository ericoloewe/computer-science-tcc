\documentclass{article}
\usepackage{graphicx,url}
\usepackage{amsmath}
\usepackage{listings}
\usepackage[brazil]{babel}   
\usepackage[utf8]{inputenc}

\title{ANTEPROJETO}
\author{ÉRICO DE SOUZA LOEWE}

\date{Today}

\begin{document}

\makeatletter
\begin{titlepage}
   \vspace*{\stretch{1.0}}
   \begin{center}
      \large\textbf{
      UNIVERSIDADE FEEVALE\\
      \@title\\
      \@author\\
      TÍTULO DO TRABALHO\\
      }
      (Título Provisório)\\
      Anteprojeto de Trabalho de Conclusão\\
      \today
   \end{center}
   \vspace*{\stretch{2.0}}
\end{titlepage}
\makeatother

\section{Folha de rosto}
\newpage

\section{Resumo}
\newpage

\section{Motivação}

Sistemas de recomendação (RecSys) são implementações de softwares e técnicas, que apresentam sugestões de itens que seriam de uso de um usuário. As sugestões são de acordo com vários processos de decisão, como, que item comprar, que musica escutar ou que noticia ler. \cite{ricci2011introduction}

As recomendações personalizadas necessitam que o sistema conheça algo sobre cada usuário da base.
Todo sistema de recomendação deve desenvolver e manter um \textit{user model} ou \textit{user profile}, que por exemplo, contem as preferencias do mesmo.
A existencia de um \textit{user model} é essencial para qualquer sistema de recomendação. \cite{jannach2010recommender}

As preferencias de um usuário podem ser obtidas implicitamente atraves de um monitoramento de comportamento. Mas também, um sistema de recomendação pode obter explicitamente sua preferencia através de perguntas sobre suas preferencias.
\cite{jannach2010recommender}

Atualmente, os Spotify não consegue captar um sentimento tão bem, devido a não levar em consideração as ações que os usuários estão tomando naquele momento. 

A ideia desse projeto é que com ele, possamos evoluir a recomendação musical para algo mais voltado ao sentimento que o usuário esta sentindo no momento em que esta ouvindo a musica.

\newpage

\section{Objetivos}

\subsection{Objetivo geral}

Estudar sistemas de recomendação (RecSys). Apos, analisar o sistema de recomendação musical utilizado pelo Spotify, e realizar uma comparação com um sistema de recomendação em tempo real, desenvolvendo um \textit{plugin} que ira contemplar esse novo sistema.

\subsection{Objetivo específicos}

\begin{itemize}
\item Estudar os sistemas de recomendação (RecSys)
\item Estudar os conceitos básicos de Deep Learning
\item Estudar os sistemas de recomendação em tempo real
\item Estudar os sistemas de recomendação do Spotify
\item Desenvolver plugin para criar playlists no Spotify
\item Analisar os resultados obtidos
\end{itemize}

\newpage

\section{Metodologia}

\newpage

\section{Cronograma}

\newpage

% \bibliographystyle{sbc}
\bibliographystyle{acm}
\bibliography{references}

\end{document}
